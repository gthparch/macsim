
\chapter{SST-Macsim}

Macsim is a part of SST simulation framework~\cite{sst}.

\section{How to install SST}

Follow the instructions provided in wiki page at
the \textit{sst-simulator} GoogleCode repository~\cite{sst-google}.

\section{How to install \SIM in SST}

Check out the svn copy of \SIM in \textit{sst-top/sst/elements} and
apply a patch. 

\begin{Verbatim}
cd sst-top/sst/elements
svn co https://svn.research.cc.gatech.edu/macsim/trunk macsim
cd macsim
patch -p0 -i macsim-sst.patch
\end{Verbatim}

Then, re-run the SST build procedure.

\begin{Verbatim}
cd sst-top
./autogen.sh
./configure --prefix=/usr/local --with-boost=/usr/local --with-zoltan=/usr/local --with-parmetis=/usr/local
\end{Verbatim}

\section{How to configure the \SIM SST component}

An example SST sdl configuration file is as follows:

\begin{Verbatim}
<?xml version="1.0"?>
<sdl version="2.0"/>

<config>
  stopAtCycle=1000s
  partitioner=self
</config>

<sst>
  <component name=gpu0 type=macsimComponent.macsimComponent rank=0 >
    <params>
      <paramPath>./params.in</paramPath>
      <tracePath>./trace_file_list</tracePath>
      <outputPath>./results/</outputPath>
      <clock>1.4Ghz</clock>
    </params>
  </component>
</sst>
\end{Verbatim}

In this manner, an SST simulation configuration file can declare multiple 
instances of \SIM as well as define what traces are run on each \SIM instance.

\section{How to run a \SIM simulation in SST}

Ensure SST and \SIM components are compiled and/or installed.  
Ensure that the paths and contents of both SST configuration sdl file and
\SIM \textit{params.in} configuration file are correct. Start the SST simulation 
either standalone or through MPI.

\begin{Verbatim}
sst.x [sdl-file]
\end{Verbatim}
or
\begin{Verbatim}
mpirun -np [x] sst.x [sdl-file]
\end{Verbatim}

% LocalWords:  Macsim
