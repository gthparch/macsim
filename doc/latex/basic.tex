

%%%%%%%%%%%%%%%%%%%%%%%%%%%%%%%%%%%%%%%%%%%%%%%%%%%%%%%%%%%%%%%%%%%%%%%%
\chapter{Getting Started}
%%%%%%%%%%%%%%%%%%%%%%%%%%%%%%%%%%%%%%%%%%%%%%%%%%%%%%%%%%%%%%%%%%%%%%%%


In this chapter, we provide the basic instruction to install
\SIM. Then, we explain how to run \SIM and provide tips to analyze
simulation outcomes.


%%%%%%%%%%%%%%%%%%%%%%%%%%%%%%%%%%%%%%%%%%%%%%%%%%%%%%%%%%%%%%%%%%%%%%%%
\section{Wiki and Other Supports}
%%%%%%%%%%%%%%%%%%%%%%%%%%%%%%%%%%%%%%%%%%%%%%%%%%%%%%%%%%%%%%%%%%%%%%%%

We manage the google project page in the following url:
\urlc{http://code.google.com/p/macsim}. In this page, we will provide
detailed information regarding \SIM, including issue/bug tracking,
sample traces, and other instructions. If you have a question or find
a bug, you can utilize the project page.



\section{Installation}
\label{sec:installation}

%%%%%%%%%%%%%%%%%%%%%%%%%%%%%%%%%%%%%%%%%%%%%%%%%%%%%%%%%%%%%%%%%%%%%%%%
\subsection{Download}
%%%%%%%%%%%%%%%%%%%%%%%%%%%%%%%%%%%%%%%%%%%%%%%%%%%%%%%%%%%%%%%%%%%%%%%%

\SIM source code is maintained using the subversion. 
You can check out the \SIM copy by


\begin{Verbatim}
svn co https://svn.research.cc.gatech.edu/macsim/trunk macsim-readonly --username readonly
\end{Verbatim}

\noindent
Initially, read-only permission is given to users. If you want to
contribute to \SIM, you need to send an email to
\href{mailto:macsim-dev@googlegroups.com}{macsim-dev@googlegroups.com}
to acquire such permission.




%%%%%%%%%%%%%%%%%%%%%%%%%%%%%%%%%%%%%%%%%%%%%%%%%%%%%%%%%%%%%%%%%%%%%%%%
\subsection{Build Requirement}
%%%%%%%%%%%%%%%%%%%%%%%%%%%%%%%%%%%%%%%%%%%%%%%%%%%%%%%%%%%%%%%%%%%%%%%%

To build \SIM propery, it requires following supports:

\begingroup
\renewcommand\descriptionlabel[1]{\textit{\hspace\labelsep{#1}}}
%\renewcommand\descriptionlabel[1]{\hspace\labelsep\cs{#1}}
\begin{description}\firmlist
  \item[Operating System] Currently, we only support linux
    distributions. Tested systems are as follows: \Verb+Ubuntu+, \Verb+Redhat (TODO)+.
  \item[Compiler] Any compiler that supports the C++0x (or C++11)
    standard library. Currently, we tested follwing compilers: \Verb+gcc+, \Verb+icc (TODO)+.
  \item[Autotools] - You need to have autotools (automake, autoconf,
    ...) version 2.65 or higher. You can install autotools by
\begin{Verbatim}
Ubuntu: apt-get install autotools-dev automake autoconf
Redhat: todo
\end{Verbatim}
  \item[Libraries] - gzip
\begin{Verbatim}
Ubuntu: apt-get install zlib1g-dev
\end{Verbatim}
\end{description}
\endgroup




\ignore{
\subsection{Directory Structure}

This section explains the directory structure of \SIM simulator.

\smallskip
\begin{lstlisting}
macsim/
  tag/ branch/ trunk/
\end{lstlisting}
\smallskip

\textit{Tag} directory has tagged version of \SIM
simulators. \textit{Branch} directory is for diverged \SIM, which is
currently empty. \textit{Trunk} directory is current working directory
for \SIM.

Following is more detailed information about \textit{Trunk} directory.

\smallskip
\begin{lstlisting}
trunk/
  bin/ def/ doc/ params/ scripts/ src/ tools/
\end{lstlisting}
\smallskip

\textit{Bin} directory contains the \SIM binary after the building
process. \textit{def} directory has knob (Section~\ref{sec:knob}) and
statistics (Section~\ref{sec:stat}) definitions. \textit{doc} has the
documentation. \textit{scripts} includes several scripts files that
are using during the building process. \textit{src} contains all
source files. \textit{tools} has several useful tools.
\ignore{ (Section~\ref{sec:tool}) }
}





%%%%%%%%%%%%%%%%%%%%%%%%%%%%%%%%%%%%%%%%%%%%%%%%%%%%%%%%%%%%%%%%%%%%%%%%
\subsection{Installation}
%%%%%%%%%%%%%%%%%%%%%%%%%%%%%%%%%%%%%%%%%%%%%%%%%%%%%%%%%%%%%%%%%%%%%%%%

The GNU Autotools (aclocal, automake, autoconf, ...) have been used to
build \SIM. After initial check-out of the \SIM copy, following
commands are necessary (You can refer to \Verb+INSTALL+ file as well).


\begin{Verbatim}
aclocal 
automake 
--add-missing 
autoconf 
./configure 
make
\end{Verbatim}

\noindent
You can combine above commands in a line:

\begin{Verbatim}
aclocal && automake --add-missing && autoconf && ./configure && make
\end{Verbatim}

\noindent
We provide autogen.sh script file to simplify the building process.

\begin{Verbatim}
./autogen.sh
make
\end{Verbatim}

\noindent
Then, the binary \bin will be generated in the \Verb+trunk/bin/+
directory.





%%%%%%%%%%%%%%%%%%%%%%%%%%%%%%%%%%%%%%%%%%%%%%%%%%%%%%%%%%%%%%%%%%%%%%%%
\subsection{Build Types}
%%%%%%%%%%%%%%%%%%%%%%%%%%%%%%%%%%%%%%%%%%%%%%%%%%%%%%%%%%%%%%%%%%%%%%%%

We provide three different build types: optimized, debug, and gprof
versions. Based on the type, we provide different compiler flags. For
example,

\begin{itemize}
  \item opt : default, optimized version (-O3 flag)
  \item dbg : debug version (-g3 flag)
  \item gpf : gprof version (-pg flag)
\end{itemize}

\noindent
To build a certain type, you need to specify the option
after \textit{make} command. For example,

\begin{Verbatim}
make [opt]
make dbg
make gpf
\end{Verbatim}





%%%%%%%%%%%%%%%%%%%%%%%%%%%%%%%%%%%%%%%%%%%%%%%%%%%%%%%%%%%%%%%%%%%%%%%%
\section{How To Run \SIM}
\label{sec:run}
%%%%%%%%%%%%%%%%%%%%%%%%%%%%%%%%%%%%%%%%%%%%%%%%%%%%%%%%%%%%%%%%%%%%%%%%

To run \bin binary, two additional files are required in
the same directory.

\begin{itemize}
  \item params.in - defines architectural parameters that will
  overwrite the default value.

  \item trace\_file\_list - defines the number of traces to run and
  the path of each trace
\end{itemize}


We provide several pre-defined architecture parameters in
\Verb+macsim-top/trunk/params+ directory. Table~\ref{table:param}
shows the list of pre-defined parameter files. If you want to run \SIM
with a certain configuration, you need to copy the corresponding
parameter file to \Verb+bin+ directory and rename it as
\Verb+params.in+. For example, if you want to use NVIDIA's
Fermi~\cite{fermi} GPU architecture, the following command would be
sufficient:

\begin{Verbatim}
cp params/params_gtx465 bin/params.in
\end{Verbatim}


\begin{table}
\begin{footnotesize}
\begin{center}
\caption{Parameter Templates}
\label{table:param}
\begin{tabular}{|l|l|l|}
\hline
File Name              & Description & Architecture                         \\ \hline \hline
params\_8800gt         &             & NVIDIA GeForce 8800 GT (G80)         \\
params\_gtx280         &             & NVIDIA GeForce GTX 280 (GT200)       \\
params\_gtx465         &             & NVIDIA GeForce GTX 465 (Fermi)       \\
params\_x86            &             & Intel's Sandy Bridge (CPU part only) \\
params\_hetero\_4c\_4g &             & Intel's Sandy Bridge (CPU + GPU)     \\ \hline
\end{tabular}
\end{center}
\end{footnotesize}
\end{table}


To specify which trace to run, you need to edit
\Verb+trace_file_list+. Following is the content of a sample
\Verb+trace_file_list+.

\begin{Verbatim}
2 <-- number of traces
/trace/ptx/cuda2.2/FastWalshTransform/kernel_config.txt <-- trace 1 path
/trace/ptx/cuda2.2/BlackScholes/kernel_config.txt <-- trace 2 path
\end{Verbatim}

\noindent
The first line is the number of traces that \SIM will run and
following lines are the path of each traces. Each top level trace
configuration file contains various information. Following is the
content of a top-level trace information file.


\begin{Verbatim}
1 x86
0 0
\end{Verbatim}

\noindent
In the first line, the first number indicates the number of threads in
the application and second literal specifies the type of the
application. Trailing lines are the information of each thread
({thread id}, {thread starting point in terms of the instruction count
  of the main thread (thread 0)}).


We provide several sample trace files in \urlc{todo}, but these samples
may not be enough for full simulation. If you want to create your own
traces, we provide the detailed information in
Chapter~\ref{sec:trace}.


%%%%%%%%%%%%%%%%%%%%%%%%%%%%%%%%%%%%%%%%%%%%%%%%%%%%%%%%%%%%%%%%%%%%%%%%
\subsection{Run}
%%%%%%%%%%%%%%%%%%%%%%%%%%%%%%%%%%%%%%%%%%%%%%%%%%%%%%%%%%%%%%%%%%%%%%%%

The following is the sample command lines to run \SIM simulator.

\begin{Verbatim}
./macsim
\end{Verbatim}



%%%%%%%%%%%%%%%%%%%%%%%%%%%%%%%%%%%%%%%%%%%%%%%%%%%%%%%%%%%%%%%%%%%%%%%%
\section{How to Analyze Output}
%%%%%%%%%%%%%%%%%%%%%%%%%%%%%%%%%%%%%%%%%%%%%%%%%%%%%%%%%%%%%%%%%%%%%%%%

\SIM generates several output files once the simulation is succesfully
completed. Here are the list of output files.

\begin{Verbatim}
params.out
statfile1.stat.out
statfile2.stat.out
...
statfilen.stat.out
\end{Verbatim}

\noindent
\Verb+params.out+ enumerates all configurations with their
values. Other files are statistics outputs, which is mostly the number
of occurrence for architectural events during the simulation. Each
stat out file consists of following lines:

\begin{Verbatim}
STAT     Raw_counter_value     Value_per_type
\end{Verbatim}

\noindent
, where \Verb+STAT+ is the name of stat, \Verb+Raw_counter_value+ is
the raw value of the counter, and \Verb+Value_per_type+ is the value
based on the each stat type. There are three types of statistics:

\begin{itemize}

  \item COUNT - for counting the number of occurances of an event. Eg. number
  of cache hits. 

  \item RATIO - for calculating the ratio of number of occurances of one event
  over another. Eg. (number of cache hits / number of cache accesses) i.e.,
  cache hit ratio.

  \item DIST - for defining a group of related events and calculating the
  number of occurances of each event in the as a percent of the sum of the
  number of occurances of all events in the group.  Eg. If we want to know what
  percent of L1 data cache accesses (in a 2-level hierarchy) resulted in L1
  hits, L2 hits or memory accesses, we should define a distribution consisting
  on three events - L1 hits, L2 hits and L2 misses  - and update the
  counter for each event correctly. 
\end{itemize}


\noindent
We sometimes need to measure a certain event per core, so we further
categorize statistics into either global (common counter for all
cores) or per core (each core has its own counter). The suffix of all
per core stats is \Verb+_CORE_#+. Chapter~\ref{sec:stat} details
regarding statistics.


\begin{table}[htb]
\begin{footnotesize}
\begin{center}
\caption{Important Stats.} 
\label{table:stats}
\begin{tabular}{|l||l|c|l|}
\hline 
STAT & Description & Per-Core & Filename \\ \hline \hline
INST\_COUNT\_TOT            & \# of instructions                                    &      & general.stat.out \\ \hline 
INST\_COUNT\_CORE\_[0-11]   & \# of instructions in only the specificed core [0-11] & core & general.stat.out \\ \hline 
CYC\_COUNT\_TOT             & simulated cycles                                      &      & general.stat.out \\ \hline 
CYC\_COUNT\_CORE\_[0-11]    & simulated cycles in only the specificed core [0-11]   &      & general.stat.out \\ \hline 
CYC\_COUNT\_X86             & simulated cycles for x86 only                         &      & general.stat.out \\ \hline 
CYC\_COUNT\_PTX             & simulated cycles for ptx only                         &      & general.stat.out \\ \hline \hline 
                            & \# of fp instructions                                 &      &                  \\ \hline 
                            & \# of int instructions                                &      &                  \\ \hline 
                            & \# of load instructions                               &      &                  \\ \hline  
                            & \# of store instructions                              &      &                  \\ \hline  
BP\_ON\_PATH\_CORRECT       & \# of correctly predicted branches (DIST)             & core & bp.stat.def      \\ \hline  
BP\_ON\_PATH\_MISPREDICT    & \# of mis-predicted branches (DIST)                   & core & bp.stat.def      \\ \hline  
BP\_ON\_PATH\_MISFETCHT     & \# of mis-fetch branches (BTB MISS)(DIST)             & core & bp.stat.def      \\ \hline  
ICACHE\_HIT, ICACHE\_MISS   & \# of I-cache hitt,miss (DIST)                        &      & memory.stat.def  \\ \hline  
L[1-3]\_HIT\_CPU            & \# of l[1-3]cache hits from CPU                       &      & memory.stat.def  \\ \hline 
L[1-3]\_HIT\_GPU            & \# of l[1-3]cache hits from GPU                       &      & memory.stat.def  \\ \hline 
L[1-3]\_MISS\_CPU           & \# of l[1-3]cache misses from CPU                     &      & memory.stat.def  \\ \hline 
L[1-3]\_MISS\_GPU           & \# of l[1-3]cache misses from GPU                     &      & memory.stat.def  \\ \hline  \hline 
AVG\_MEMORY\_LATENCY        & average memory latency                                &      & memory.stat.def  \\ \hline \hline 
TOTAL\_DRAM                 & \# of DRAM accesses                                   &      & memory.stat.def  \\ \hline  
TOTAL\_DRAM\_READ           & \# of DRAM reads                                      &      & memory.stat.def  \\ \hline  
TOTAL\_DRAM\_WB             & \# of DRAM write backs                                &      & memory.stat.def  \\ \hline  
                            & \# of register reads                                  &      &                  \\ \hline  
                            & \# of register writes                                 &      &                  \\ \hline   \hline 
COAL\_INST, UNCOAL\_INST    & coalesced/uncoalesced mem requests (DIST)             &      & memory.stat.def  \\ \hline 
\end{tabular}
\end{center}
\end{footnotesize}
\end{table} 


You can 


\begin{Verbatim}
IPC = INST_COUNT_TOT / CYC_COUNT_TOT
CPI = CYC_COUNT_TOT / INST_COUNT_TOT
\end{Verbatim}

\begin{Verbatim}
IPC_1 = INST_COUNT_CORE_1 / CYC_COUNT_CORE_1
IPC_2 = INST_COUNT_CORE_2 / CYC_COUNT_CORE_2
...
IPC = harmonic_mean(IPC_1, IPC_2, ..., IPC_n)

\end{Verbatim}


Table~\ref{table:stats} shows important stats.



\section{SST-Macsim}

Macsim is a part of SST simulation framework~\cite{sst}. This section
is intended for SST users who want to use \SIM for GPU simulations.

\subsection{How to install SST}

Follow the instructions provided in wiki page at the
\textit{sst-simulator} GoogleCode repository~\cite{sst-google} at
\urlc{http://code.google.com/p/sst-simulator}.

\subsection{How to install \SIM in SST}

Check out the svn copy of \SIM in \textit{sst-top/sst/elements} and
apply a patch. 

\begin{Verbatim}
cd sst-top/sst/elements
svn co https://svn.research.cc.gatech.edu/macsim/trunk macsim
cd macsim
patch -p0 -i macsim-sst.patch
\end{Verbatim}

Then, re-run the SST build procedure.

\begin{Verbatim}
cd sst-top
./autogen.sh
./configure --prefix=/usr/local --with-boost=/usr/local --with-zoltan=/usr/local --with-parmetis=/usr/local
\end{Verbatim}

\subsection{How to configure the \SIM SST component}

An example SST sdl configuration file is as follows:

\begin{Verbatim}
<?xml version="1.0"?>
<sdl version="2.0"/>

<config>
  stopAtCycle=1000s
  partitioner=self
</config>

<sst>
  <component name=gpu0 type=macsimComponent.macsimComponent rank=0 >
    <params>
      <paramPath>./params.in</paramPath>
      <tracePath>./trace_file_list</tracePath>
      <outputPath>./results/</outputPath>
      <clock>1.4Ghz</clock>
    </params>
  </component>
</sst>
\end{Verbatim}

In this manner, an SST simulation configuration file can declare multiple 
instances of \SIM as well as define what traces are run on each \SIM instance.

\subsection{How to run a \SIM simulation in SST}

Ensure SST and \SIM components are compiled and/or installed.  
Ensure that the paths and contents of both SST configuration sdl file and
\SIM \textit{params.in} configuration file are correct. Start the SST simulation 
either standalone or through MPI.

\begin{Verbatim}
sst.x [sdl-file]
\end{Verbatim}
or
\begin{Verbatim}
mpirun -np [x] sst.x [sdl-file]
\end{Verbatim}



%%%%%%%%%%%%%%%%%%%%%%%%%%%%%%%%%%%%%%%%%%%%%%%%%%%%%%%%%%%%%%%%%%%%%%%%
\section{How to Get Memory Traces Using \SIM}
%%%%%%%%%%%%%%%%%%%%%%%%%%%%%%%%%%%%%%%%%%%%%%%%%%%%%%%%%%%%%%%%%%%%%%%%

\TODO{TODO}










