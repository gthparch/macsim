
\clearpage
\section{Trace Generation}

Explain how to generate traces and the format of the trace.

\subsection{CPU (X86) Traces}

Describe pin

\subsection{GPU (PTX) Traces}

GPU traces are generated using Ocelot~\cite{}, which is a dynamic compilation
framework for heterogenous system. 

\subsubsection{Installing Ocelot}

Ocelot can be installed either using ubuntu packages provided by the official
page or via its SVN repository. To install the latest version, check out 
source files using the following command:

\smallskip
\begin{lstlisting}
svn checkout http://gpuocelot.googlecode.com/svn/trunk/ gpuocelot
\end{lstlisting}
\smallskip

After checking out source files, you can build Ocelot and trace generator libraries using the following commands:

\smallskip
\begin{lstlisting}
cd gpuocelot/ocelot; sudo ./build.py --install
cd gpuocelot/trace-generators; libtoolize; aclocal; autoconf; automake; ./configure; 
make; sudo make install
\end{lstlisting}
\smallskip


\subsubsection{Generating Traces}

CUDA executables which are targeted to generate traces should be linked against
libocelot.so and libocelotTrace.so.

In order to generate traces, the following four environment variables need to be set.

\begin{itemize}\itemsep2pt
\item TRACE\_PATH : directory to store generated traces. If not specified, current directory is used by default.
\item TRACE\_NAME : prefix name for generated traces. If not specified, "Trace" is used by default.
\item KERNEL\_INFO\_PATH : file that contains kernel information (must be specified)
\item COMPUTE\_VERSION : compute capability (must be specified)
\end{itemize}

The following shows an example of how to set up the environment variables.

\smallskip
\begin{lstlisting}
export TRACE_PATH="/storage/traces/"     # Create a trace directory in the /storage/traces
export KERNEL_INFO_PATH="kernel_info"    # kernel_info has the kernel information
export COMPUTE_VERSION="2.0"             # Calculate occupancy based on compute capability 2.0
\end{lstlisting}
\smallskip

The kernel information file (kernel\_info) should contain the following
information: kernel name, register usage (per thread), and shared memory usage
(per thread).

\smallskip
\begin{lstlisting}
_Z9Memcpy_SWPfS_i 14 52 
\end{lstlisting}
\smallskip

Running CUDA application creates a directory with the kernel name, where traces 
are generated, and kernel\_config.txt which is a configuration used for MacSim.

\smallskip
\begin{lstlisting}
drwxr-xr-x   2 anonymous group 69632 Sep 00 22:03 _Z9Memcpy_SWPfS_i_0
-rw-r--r--   1 anonymous group    66 Sep 09 22:29 kernel_config.txt
\end{lstlisting}
\smallskip


\subsection{Trace Format}

Describe the trace format.




% LocalWords:  GPU PTX gpuocelot
